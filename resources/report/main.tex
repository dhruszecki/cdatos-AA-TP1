\documentclass[10pt]{article}

\usepackage[utf8]{inputenc}
\usepackage{mathtools}
\usepackage{cite}
\usepackage{graphicx}


\author{
  Cepeda, Diego\\
  \texttt{fdecepeda@gmail.com}
  \and
  Guizzardi, Santiago\\
  \texttt{santiagoguizzardi@gmail.com}
  \and
  Hruszecki, Darío\\
  \texttt{dario.hruszecki@gmail.com}
}

\title{
Aprendizaje Automático\\
Trabajo práctico Nro 1\\
}

\date{Junio 2020}

\begin{document}

\begin{titlepage}
   \begin{center}
       \vspace*{1cm}

       \textbf{Aprendizaje Automático}

       \vspace{0.5cm}
        Trabajo práctico Nro 1
            
       \vspace{1.5cm}

       \textbf{
       	  Cepeda, Diego\\
		  Guizzardi, Santiago\\
		  Hruszecki, Darío\\
	    }

       \vfill
            
            
       \vspace{0.8cm}
     
       \includegraphics[width=0.4\textwidth]{images/320px-Logo_exactas.svg}

       \vspace{0.8cm}

        Maestría en Explotación de Datos y Descubrimiento de Conocimiento\\
		Facultad de Ciencias Exactas y Naturales - Facultad de Ingeniería\\
		Universidad de Buenos Aires\\
		Junio 2020            
   \end{center}

\end{titlepage}


\maketitle

\section{Resúmen}

\textbf{un resumen (del estilo de un artículo científico de no más de 200 palabras)}

\section{Introduction}
\textbf{una introducción en donde, entre otros, conste el objetivo del trabajo y una explicación de cómo está organizado el resto del documento}

\section{Datos}

\textbf{una sección de datos, en donde se describan los datos utilizados y sus particularidades}

\section{Metodología}

\textbf{una sección de metodología, en donde se describan las metodologías utilizadas (sobre datos y sobre algoritmos)}

\section{Resultados}

\textbf{una sección resultados, que incluya los resultados y su análisis}

\section{Conclusiones}

\textbf{Por tratarse de un trabajo de investigación netamente práctico, las conclusiones deben ser la resultante de la elaboración de las pruebas realizadas. La información obtenida de referencias externas puede y debe ser tomada como insumo, pero no como conclusión \cite{Zychlinski:1}}

\newpage
\nocite{*} 

\bibliography{biblio} 
\bibliographystyle{plain}

\end{document}
